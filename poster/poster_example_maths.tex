%%% ---- LaTeX template for a simple A2 poster
%%%     landscape orientation, with a full-width header
%%%         and body material in columns
%----------------------------------------------
\documentclass[10pt,a4paper,landscape]{article}
%%%  magnify to A2 when printing the final version
%%%  portrait orientation is default
%----------------------------------------------
\usepackage[noheadfoot,width=277mm,height=200mm]{geometry}
%%%  use all the paper -- leave no head- or foot-space
\pagestyle{empty}  %%% suppress page numbering
%----------------------------------------------
%\usepackage{graphicx,multicol,color} %%% default CM font
\usepackage{graphicx,multicol,color,mathpazo} %% mathpazo font
% \usepackage{graphicx,multicol,color,mathptmx}  %% times roman font
%----------------------------------------------
%%  this is for using an image as background . . .
\usepackage{everyshi,eso-pic,calc,ifthen,wallpaper}
%----------------------------------------------
%%%  homebrew colours for text, backgrounds, etc
%%%                    .... by mixing red, green and blue
\definecolor{cola}{rgb}{.9,.9,1}
\definecolor{colb}{rgb}{.9,1,.9}
\definecolor{colc}{rgb}{1,.9,.9}
%------------------------------------------------------------
%% boxes to put stuff in ......
%%
%%   (1) transparent boxes --- the background shows through
%%			   #1 = width as fraction of what's available
%%                         #2 = contents (text, formula, picture)
%%                               ... tho pictures are always opaque
%% ---- box for headers --- width = fraction of textwidth 
\newcommand\BoX[2]{\begin{minipage}{#1\textwidth}#2\end{minipage}}
%% ---- box centred in a column --- width = fraction of columnwidth
\newcommand\BOX[2]{\begin{center}
   \begin{minipage}{#1\columnwidth}#2\end{minipage}\end{center}\vfill}
%%
%%   (2) opaque boxes with coloured background and outline frame
%%                         #1 = width as fraction of what's available
%%                         #2 = contents (text, formula, picture)
%%                         #3 = background colour
%%                         #4 = frame colour
\setlength\fboxrule{2pt} %% = width of frame lines round boxes
\setlength\fboxsep{5pt}  %% = spacing round box contents
%% ---- box for headers --- width = fraction of textwidth 
\newcommand\cBoX[4]{\fcolorbox{#4}{#3}{%
	\begin{minipage}{#1\textwidth}#2\end{minipage}}}
%% ---- box centred in a column --- width = fraction of columnwidth
\newcommand\cBOX[4]{\begin{center}\fcolorbox{#4}{#3}{%
 \begin{minipage}{#1\columnwidth}#2\end{minipage}}\end{center}\vfill}
%--------------------------------------------------------------------
%% for e.g. picture and text side-by-side in a column
%%   n.b. this command must be inside a \BOX or a \cBOX
\newcommand\sidebyside[2]{\BoX{.46}{#1}\hfill\BoX{.46}{#2}}
%--------------------------------------------------------------------
\setlength\columnsep{8mm}  %%%% column separation
%------------------------------------------------------------
\begin{document}
%------------------------------------------------------------
%%   remove this line if the background is an image
\pagecolor{cola}
%------------------------------------------------------------
%%%   ... and uncomment this line for image.jpg as a background
% \ThisCenterWallPaper{1.0}{image.jpg}
%%     .... cover the page by adjusting the value 1.0
%------------------------------------------------------------
%%     first the headers - title, logo, etc across the top
%%%       ... use \BoX and/or \cBoX here
\cBoX{.4}{\color{blue}\centering\Huge\textsc{Farmer and
Trespasser}}{colb}{red}
\hfill
\BoX{.16}{\centering\textit{\lq\lq Get orf moi laaaand!\rq\rq}}
\hfill
\BoX{.2}{\centering\includegraphics[width=0.6\textwidth]{logo.jpg}}
%------------------------------------------------------------
\vspace{0.5in}  %%%%   adjust the space between headers and body
%------------------------------------------------------------
%%%  now the body material in 3 columns of stacked boxes
%%      ... use \BOX and/or \cBOX here
\begin{multicols*}{3}
%%%--------------------------------------------
%%%   to get more in, try say 4 columns with smaller-sized type
%%%  BUT -- for easy reading 
%%%               use no more than about 66 characters per line
%%%%   .... and choose a size of type that 
%%%%              looks OK expanded from A4 to A2
% \tiny
\small
%%%---------------------------------------------------------
\BOX{.9}{\subsection*{problem}Farmer Palmer, lying stealthily on
the (flat) ground, shoots at
a trespasser who has climbed to the top of a tree of height $h$ at
distance $L$ away.\par If bullets leave his gun with speed $v$ then find
an equation determining the required angle of projection $\alpha$.\par
Show that $\tan\alpha=v^2/gL$ for maximum range, and that then the
bullets' initial velocity bisects the angle between the vertical and the
straight line joining gun and trespasser.}
\BOX{0.9}{\subsection*{diagram}
\includegraphics[width=\textwidth]{diagram.jpg}}
%%%%  new column -------------------------------------
\columnbreak
\BOX{0.9}{\subsection*{derivation}
integrating N2 gives
\[ x(t)=vt\cos\alpha\qquad y(t)=vt\sin\alpha-gt^2/2 \]
when putting \[ x(T)=L\qquad\mbox{and}\quad y(T)=h \] and eliminating
time of flight $T$ --- with the use of 
\[ \tan=\sin/\cos\qquad\mbox{and}\quad\sin^2+\cos^2=1 \] --- leads to
the \textbf{key equation}.}
\cBOX{0.9}{\subsection*{key equation}
\[  h=-\frac{gL^2}{2v^2}(1+\tan^2\alpha)+L\tan\alpha  \]
\medskip\hfill--- which is quadratic for $\tan\alpha$.}{colb}{colc}
\BOX{0.9}{\subsection*{maximum range}
in range there are two real solutions for $\tan\alpha$ and out of range
the solutions are complex --- so at \textbf{maximum range} they coincide
--- when
\[ \tan\alpha=-\frac{L}{2(-gL^2/2v^2)}=\frac{v^2}{gL} \]
--- remembering that when $ax^2+bx+c=0$ has coincident roots
they are given by $x=-b/2a$.}
%%%%  new column -------------------------------------
\columnbreak
\BOX{0.9}{\subsection*{angles}
let $\beta$ be the angle between the ground and the straight line
joining gun and trespasser --- then $\tan\beta=h/L$, which the
\textbf{key equation} gives as
\[ \tan\beta=\frac hL=-\frac{gL}{2v^2}(1+\tan^2\alpha)+\tan\alpha \]
and where at \textbf{maximum range} we have \[ gL/v^2=1/\tan\alpha \]
--- hence
\[ \tan\beta=\frac{\tan^2\alpha-1}{2\tan\alpha}=-\cot2\alpha \]
and so $\beta=2\alpha-\pi/2$ or
\[ \pi/2-\alpha=\alpha-\beta \]
as required}
\BOX{0.9}{\subsection*{\dots\ and finally}
why does angle $\alpha$ at \textbf{maximum range} depend on horizontal
distance $L$ and not on vertical height $h$?}
\BOX{0.9}{\subsection*{references}
http://en.wikipedia.org/wiki/Farmer\textunderscore Palmer}
\setlength\fboxrule{1pt}  %%%% a thinner frame here
\cBOX{0.9}{\centering%
RC Johnson\quad\textsl{Maths Dept}\quad\today}{colb}{colc}
%----------------------
\end{multicols*}
%%  ---------------------------------------------------
\end{document}
%%%%%%%%%%%%%%%%------------------------------
