%!TEX root = report.tex

In 1905 a young Albert Einstein published a paper entitled ``On the Electrodynamics of Moving Bodies''. This paper was the beginning of the revolution that relativistic physics would bring to our understanding of nature. Einstein abandoned the notion that time and space were separate entities, and instead intimately tied them together. However, the origins of this significant change to our view of nature were founded earlier, with the advent of classical electrodynamics. 

Around 1862 James Clerk Maxwell published his equations which united electric and magnetic phenomena into one unified theory. One of the most striking results which can be derived from these equations is that light is an electromagnetic phenomenon. However, there was a problem: physicists of the time were unable to pinpoint the medium through which the light waves propagated. The solution to this problem was suggested by Maxwell; light propagated through the ``luminiferous \ae ther'', a substance pervading the whole universe which light waves would perturb as they travelled. This idea would plague physics for the next 40 or so years, with successive experiments finding evidence which would both support and contradict the concept of the \ae ther.

Einstein's great insight in 1905 was to do away with the \ae ther and to suggest that the speed of light is the same in all inertial frames. This was a radical idea, going against centuries of established science at the time. In doing so Einstein vastly simplified the calculations required to fit with all observational data at that point \cite{cheng}. 

On coming up with this idea Einstein had to rethink commonly held notions of time and space. Due to the constant nature of the speed of light in all frames, time had to proceed differentially for observers moving at different speeds. From here the absolute notions of length had to be discarded, and instead lengths would be observed differently by different observers. These were the main changes brought by the special theory of relativity \cite{cheng}. 

Special relativity had a major flaw unfortunately: it did not account for gravity and thus could not be a full explanation of the mechanics of moving bodies. Once Einstein had published special relativity he began to work on the problem of integrating gravity with his new views of physics. One of the most important features of special relativity was that the physical laws of electromagnetism and mechanics had to be invariant under transformation from one inertial frame to another. Einstein realised this coordinate free description of physical laws was how any physical law should be written, so as to not prefer any frame of reference. He referred to this as the principle of general covariance and used it as a guide while inventing general relativity \cite{history}. 

Guided by his desire for the physical laws to be coordinate independent, Einstein developed his theory of gravity using the mathematical language of tensors. This theory would relate the gravitational field at a point to the curvature of the spacetime as caused by nearby mass. Describing the fundamentals of this theory will be the topic of chapter 2 of this document.

The claims made by Einstein's theories are drastic and counterintuitive. Hence the scientific community would only accept them with great supporting evidence. Much of this document will be exploring that evidence, and seeing how we can produce experiments to verify Einstein's bold claims.

Finally, the applications to technology that relativistic physics have produced are fairly substantial, mostly in the implementation of the Global Positioning System. Without our understanding of relativity the GPS would not function to the level of accuracy that it does. The effects of relativity on the system will be examined and some calculations of the corrections due to relativistic effects will be demonstrated.

Throughout this text we will take the speed of light \(c = 1\) in order to make formulas slightly clearer. If terms involving the speed of light are introduced they will be explicitly mentioned in the text.  