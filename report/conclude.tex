%!TEX root = report.tex

General relativity continues to stand up to great number of experimental tests, many more than we have had time to survey in this project \cite{lrr-2006-3}. The direct observational evidence and direct uses of general relativity in technology we use every day certainly point to the validity of the theory in most gravitational circumstances.

However, this is not to say that general relativity is a completely proven theorem. Increasingly, observations of large scale astronomical phenomena are demanding physics which may not be contained within general relativity. An example of this is the observation of dark matter and dark energy, which may require reformulations of the Einstein equation in order to explain the gravitational fields which seem to be observed.

Additionally, there are other theories of gravitation which ever more sensitive observations will be able to test the validity of. As an example of this, we can consider \(f(R)\) gravity, where the Einstein-Hilbert action \footnote{Using this action it is possible to derive the Einstein equation \eqref{einstein-eqn} using variational principles}
\begin{equation}
	S = \int \sqrt{- |g_{\mu \nu}|} \frac{R}{16 \pi G} d^4 x
\end{equation}
is altered to include a new scalar function \(f(R)\) \cite{li, lrr-2010-3}
\begin{equation}
	S = \int \sqrt{- |g_{\mu \nu}|} \frac{R + f(R)}{16 \pi G} d^4 x
\end{equation}
which would make particles cluster together quicker, producing the effects of dark matter gravitationally. There are many theories such as these, and putting a bound on the possibilities of alternate theories of gravitation is an important effort for experimental physicists \cite{lrr-2006-3}. 

On the whole though, the predictions of general relativity fit with almost all observational data found. The fundamental ideas of general relativity, that mass curves spacetime which then creates gravitation is not in doubt at all. Also, the coordinate independent physics using manifolds and tensors that general relativity introduced are very deeply embedded into how fundamental physics works, and will certainly not be discarded any time soon. Thus, I feel it fair to say that general relativity has been, and will remain, an excellent description of gravitation in our universe.